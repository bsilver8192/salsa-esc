\documentclass[fleqn]{article}
\usepackage[paperheight=11in,paperwidth=8.5in,margin=0.5in]{geometry}
\usepackage{fullpage}
\usepackage{enumitem}
\usepackage{listings,lstautogobble}
\usepackage{amsmath}
\usepackage{tikz}
\usetikzlibrary{arrows,automata,positioning}
\usepackage[normalem]{ulem}
\usepackage{scrextend}
\usepackage{textcomp}
\usepackage[T1]{fontenc}
\usepackage{lmodern}

% Left-justify {align}.
\setlength{\mathindent}{0pt}

\begin{document}
\begin{flushleft}

\section{Introduction}
Everything uses the sign conventions and symbol meanings from
Mevey2009.

http://krex.k-state.edu/dspace/bitstream/handle/2097/1507/JamesMevey2009.pdf
http://www.icrepq.com/PONENCIAS/4.320.ANDRADA.pdf

All of the controls is going to be in terms of electrical rotations. Conversions
to/from mechanical rotations will happen outside.

\section{Model}
The state vector is defined as:
\[x =
  \begin{bmatrix}
    i_A & i_B & i_C & \theta & \omega \\
  \end{bmatrix}
^T\]
\\
The input vector is defined as:
\[u =
  \begin{bmatrix}
    v_A & v_B & v_C \\
  \end{bmatrix}
^T\]
\\
The input vector uses the line-neutral voltages, not the terminal voltages.
TODO(Brian): Would it be a better idea to dump the phase interference matrix
in here somewhere early and use terminal voltages for everything else? Probably
not, because the other equations want line-neutral ones too.
\\
The output vector is defined as:
\[y =
  \begin{bmatrix}
    i_A & i_B & i_C & \theta \\
  \end{bmatrix}
^T\]
\\
The common simplifications around ``synchronous inductance'' don't apply because
our motor isn't sinusoidal.
Keeping the individual phases split out and not applying a Clarke transform is
deliberate. The coupling between the phases with a nonsinusoidal motor is not
describable in that format. In particular, the common approach of decoupling
$i_d$ and $i_q$ relies on the synchronous inductance, which as discussed doesn't
work.
\\
Equation (B.13), for reference (for general windings, ignoring leakage):
\[
  \begin{bmatrix}
    v_A \\
    v_B \\
    v_C \\
  \end{bmatrix}
  =
  \begin{bmatrix}
    R &   &   \\
      & R &   \\
      &   & R \\
  \end{bmatrix}
  \begin{bmatrix}
    i_A \\
    i_B \\
    i_C \\
  \end{bmatrix}
  +
  \begin{bmatrix}
    L - M &       &       \\
          & L - M &       \\
          &       & L - M \\
  \end{bmatrix}
  \frac{d}{dt}
  \begin{bmatrix}
    i_A \\
    i_B \\
    i_C \\
  \end{bmatrix}
  +
  \begin{bmatrix}
    e_A \\
    e_B \\
    e_C \\
  \end{bmatrix}
\]

Solving for $\frac{d}{dt}$ of the currents:
\[
 \frac{d}{dt}
  \begin{bmatrix}
    i_A \\
    i_B \\
    i_C \\
  \end{bmatrix}
  =
  \left(
  \begin{bmatrix}
    v_A \\
    v_B \\
    v_C \\
  \end{bmatrix}
  -
  \begin{bmatrix}
    R &   &   \\
      & R &   \\
      &   & R \\
  \end{bmatrix}
  \begin{bmatrix}
    i_A \\
    i_B \\
    i_C \\
  \end{bmatrix}
  -
  \begin{bmatrix}
    e_A \\
    e_B \\
    e_C \\
  \end{bmatrix}
  \right)
  \begin{bmatrix}
    (L - M)^{-1} &              &              \\
                 & (L - M)^{-1} &              \\
                 &              & (L - M)^{-1} \\
  \end{bmatrix}
\]
Let $\alpha = (L - M)^{-1}$ to make these equations shorter.
\\
We're going to write the bEMF voltages ($e_A$ etc) as $f(\theta) \omega$.
Page 40 talks about calculating that from first principles, or we should
probably just calculate it using the 1st and 5th harmonics with coefficients
fit to measured data.

This means the per-phase torques are $f(\theta) i_A$ etc.
\\
\begin{align*}
\dot{x}(t) &=
  \begin{bmatrix}
    \dot{i_A}(t) \\
    \dot{i_B}(t) \\
    \dot{i_C}(t) \\
    \omega(t) \\
    \dot\omega(t) \\
  \end{bmatrix}
  = A(\theta) x(t) + B u(t)
\\ &=
  \begin{bmatrix}
    -R \alpha &           &           & & -f(\theta)               \\
              & -R \alpha &           & & -f(\theta - 120^{\circ}) \\
              &           & -R \alpha & & -f(\theta + 120^{\circ}) \\
              &           &           & & 1                        \\
    \frac{f(\theta)}{J} & \frac{f(\theta - 120^{\circ})}{J} &
          \frac{f(\theta + 120^{\circ})}{J} & & \\
  \end{bmatrix}
  \begin{bmatrix}
    i_A(t) \\
    i_B(t) \\
    i_C(t) \\
    \theta(t) \\
    \omega(t) \\
  \end{bmatrix}
  +
  \begin{bmatrix}
    \alpha &        &        \\
           & \alpha &        \\
           &        & \alpha \\
           &        &        \\
           &        &        \\
  \end{bmatrix}
  \begin{bmatrix}
    v_A(t) \\
    v_B(t) \\
    v_C(t) \\
  \end{bmatrix}
\end{align*}
\\
The system is linear. The only elements in the electrical model are
inductors and resistors, which are both linear.
It is also time-invariant because nothing actually varies with time.
However, A is a function of $\theta$, but that's basically just gain-scheduling.
\\
\[C =
  \begin{bmatrix}
    1 & 0 & 0 & 0 \\
    0 & 1 & 0 & 0 \\
    0 & 0 & 1 & 0 \\
    0 & 0 & 0 & 1 \\
    0 & 0 & 0 & 0 \\
  \end{bmatrix}
\]

\section{Control}
Because the goal currents will be constantly varying, and having a constant
time offset on the actual currents is a problem, we will need to do
feed-forwards.
In particular, the ``steady-state'' goal will be based on a known function of
$\theta$. Call it $g(\theta)$, so that
\[r =
  \begin{bmatrix}
    \overline{r}g(\theta) \\
    \overline{r}g(\theta - 120^{\circ}) \\
    \overline{r}g(\theta + 120^{\circ}) \\
    0 \\
    0 \\
  \end{bmatrix}
\]
where $\overline{r}$ is the actual goal current passed in.

Define the control law as
$u(t) = -K(\theta)x(t) + K_{ff}(\theta, \omega)\frac{dr}{dt}(\theta, \omega)$.

$K$ is a function of $A$, which is a function of $\theta$, so $K$ is too.

$K$ will have all 0s in the columns corresponding to $\theta$ and $\omega$ so
we don't try to drive those towards any values at all, and it doesn't matter
what the corresponding elements of $r$ are.

$K_{ff}$ should adjust $u$ so it follows changes in $r$. Because $r$ is
periodic, we can calculate $\frac{dr}{dt}$ without much trouble.
For discrete time,
\[K_{ff} =
  \begin{bmatrix}
    R T & 0   & 0   & 0 & 0 \\
    0   & R T & 0   & 0 & 0 \\
    0   & 0   & R T & 0 & 0 \\
  \end{bmatrix}
\] where $T$ is the timestep.
TODO(Brian): What does that mean in continuous time? Does it matter?
\\
Seems like we should be doing delta-u but tracking scale errors in the outputs
rather than absolute errors? Otherwise, each time the sign flips, the integrator
has to wind down and then back up the other way before it's accurate. Also,
scale errors seem a lot more likely.

\section{Commanded Torque}
To get no torque ripple,
$f(\theta)g(\theta) +
f(\theta - 120^{\circ})g(\theta - 120^{\circ}) +
f(\theta + 120^{\circ})g(\theta + 120^{\circ})$
must equal a constant.
Call it $1$ to simplify the math and make the above definition of $r$ not
require a scale factor.

Other interesting things to minimize with the chosen torque command include
switching losses, copper losses in the battery wires, copper losses in the
rotor, and iron losses.
With a fixed switching frequency, the commanded current doesn't have much
influence on switching losses.
Copper losses are proportional to $I^2$ in each part of the system.
Iron losses are composed of two parts: hysteresis and eddy currents.
Eddy losses are proportional to $\left(\frac{dB}{dt}\right)^2$
according to Andrada2004, which is proportional to
$\left(\frac{dT}{dt}\right)^2$.
Hysteresis losses are proportional to the peak current if there are no loops,
but with our motor's flux linkage loops might be interesting, although they
will increase hysteresis losses.
This is probably a bad idea, because it will heat the magnets up more than they
are likely rated for, which can lead to demagnetizing them.

Required properties: \begin{itemize}
  \item Sign of current always matches sign of flux linkage.
  \item Sign of current derivative only switches twice per rotation.
    Otherwise, you increase hysteresis losses which is a bad idea.
  \item The function is smooth.
    Otherwise, it will include a lot of harmonics and be impossible to actually
    achieve. TODO(Brian): How much does this actually matter?
\end{itemize}
Ordered list of priorities for this application: \begin{enumerate}
  \item Minimize derivative of current command.
    This means the overall current is more even, which reduces copper losses.
    This also reduces eddy losses.
  \item Minimize battery current ripple.
    This will reduce heating of the input capacitors, and also slightly
    reduce heating of the battery wires (and fuse).
  \item Minimize torque ripple.
    This reduces mechanical wear.
    Austin thinks up to 20\% is reasonable.
\end{enumerate}
Non-priorities: \begin{itemize}
  \item Be simple to calculate. We're just going to pre-calculate a table for
    all encoder positions of both the function and $\frac{dI}{d\theta}$
\end{itemize}

Commanding sinusoidal current with a sinusoidal flux linkage means
$f(\theta) = \cos(\theta)$ and
$g(\theta) = \frac{2}{3}\cos(\theta)$ (sticking the scale factors somewhere
arbitrary).
This gives
\begin{align*}
  1 =& f(\theta)g(\theta) +
  f(\theta - 120^{\circ})g(\theta - 120^{\circ}) +
  f(\theta + 120^{\circ})g(\theta + 120^{\circ}) \\
  =& \cos(\theta)\frac{2}{3}\cos(\theta) +
  \cos(\theta - 120^{\circ})\frac{2}{3}\cos(\theta - 120^{\circ}) +
  \cos(\theta + 120^{\circ})\frac{2}{3}\cos(\theta + 120^{\circ}) \\
  =& \frac{2}{3}(
  \cos^2(\theta) + \cos^2(\theta - 120^{\circ}) +
  \cos^2(\theta + 120^{\circ})) \\
  =& 1 \\
\end{align*}
Interestingly, this has non-zero battery current ripple. At medium and high
frequencies this is all supplied by the bulk capacitance on the input of the
motor controller though.

Commanding trapezoidal current with a trapezoidal flux linkage also results
in a constant torque.
The torque from each phase obviously forms a trapezoid.
If these trapezoids are aligned, at any point in time one of them is exerting
a constant torque and the other two are exerting a position-varying torque.
If the two whose torque is changing are lined up so their sum is constant,
the third phase is too, so the overall torque is also torque.
This also has the property of no battery current ripple, via similar reasoning.

For our motor,
$f(\theta) = k_1 \sin(\theta) + k_2 \sin(5 \theta)$ where $k_1$ and $k_2$ are
constants with $k_2 < k_1$ such that it only crosses zero twice per revolution.
Obviously, we could use division and get a sinusoidal or trapezoidal
per-phase torque which would have no torque ripple, but those are both
bad choices.
In particular, they both have large current derivatives and peaks.

\end{flushleft}
\end{document}
